\documentclass[12pt,letterpaper,noanswers]{exam}
\usepackage[usenames,dvipsnames,svgnames,table]{xcolor}
\usepackage[margin=0.9in]{geometry}
\renewcommand{\familydefault}{\sfdefault}
\usepackage{multicol}
\usepackage{wrapfig}
\pagestyle{head}
\header{AM 22b Class 33}{}{Apr 19: Differential equations, p. \thepage}
\runningheadrule
\headrule
\usepackage{graphicx} % more modern
\usepackage{amsmath} 
\usepackage{amssymb} 
\usepackage{hyperref}
\usepackage{tcolorbox}
\usepackage[utf8]{inputenc}
\usepackage{diagbox}
\usepackage{graphicx} 
\usepackage{enumitem}
\usepackage{tikz}
\tikzstyle{startstop} = [rectangle, rounded corners, minimum width=3cm, minimum height=1cm,text centered, draw=black]

\tikzstyle{process} = [rectangle, minimum width=3cm, minimum height=1cm, text centered, draw=black, fill=gray!20]
\tikzstyle{decision} = [ellipse, minimum width=3cm, minimum height=0.5cm, text centered, draw=black, fill=white!30]
\tikzstyle{arrow} = [thick,->,>=stealth]
\usetikzlibrary{shapes.geometric, arrows}
\pagenumbering{arabic}

\usepackage[numbered,autolinebreaks,useliterate]{mcode}

\newcommand{\mb}[1]{\underline{#1}}

\begin{document}
 \pdfpageheight 11in 
  \pdfpagewidth 8.5in




% I need to review the torus trajectories...

\begin{itemize}
\itemsep0em
\item There is a skill check today (C30, 31, 32).
\item There will be a skill check next Monday (C33, 34, 35).
\item PSet 09 is due on Thursday Apr 22nd at 6pm ET.
\item I plan to return Quiz 05 late today: Quiz 06 is on similar material (adds the divergence theorem and drops chapter 17).  It will be posted on Friday Apr 23rd.
\item Reminder: your lowest quiz score will count only half the weight of the other quizzes.
\item Contact me if it would be helpful to arrange alternate deadlines for PSet 09, Quiz 06, PSet 10.
\end{itemize}

\hrule
\vspace{0.2cm}

% partial derivatives, gradient
% local linearity, differential, directional deriv
% 2nd order partials + equations with partials

\noindent\textbf{Big picture}

We will learn how to analyze differential equations from three perspectives: using approximate solutions (slope fields + Euler's method + RK45), finding exact solutions (rarely, using separation of variables), using qualitative methods (identifying equilibrium solutions and whether they are `stable' or `unstable').

Today we will find exact solutions by hand.

\vspace{0.2cm}
\hrule
\vspace{0.2cm}

\noindent\textbf{Skill Check C33 Practice}

\begin{questions}
\item Find a family of solutions to the initial value problem $\dfrac{dx}{dt} = x^2t, x(1) = 1$.
\end{questions}

\vspace{0.2cm}
\hrule
\vspace{0.2cm}

\noindent\textbf{Skill Check C33 Practice Solution}
Separating: $\dfrac{1}{x^2} \dfrac{dx}{dt} = t$.  Integrating with respect to $t$ (and changing the variable of integration on the left hand side) $\displaystyle\int \frac{1}{x^2} dx = \int t\ dt$.  $-\dfrac{1}{x} = \frac{1}{2}t^2 + c$, so $x(t) = \dfrac{1}{t^2/2 + c}$.  $x(1) = 1$ so $\dfrac{1}{1/2 + c} = 1$.  $c = 1/2$.  We have $x(t) = \dfrac{1}{t^2/2 + 1/2}$

\vspace{0.2cm}
\hrule
\vspace{0.2cm}

\noindent\textbf{Teams}
\begin{multicols}{2}

1.  student names
\end{multicols}

\vspace{0.2cm}
\hrule
\vspace{0.2cm}


\noindent\textbf{Brief summary}
\begin{tcolorbox}
\begin{itemize}
\itemsep0em
    \item Differential equations can be used to model the evolution of quantities (for example, population) over time.
    \item For the linear differential equation $\dfrac{dx}{dt} = a(x-b)$, 
     the rate of change is proportional to the displacement of the state of the system from $b$.
     \item Near equilibrium solutions, solutions to nonlinear, autonomous (meaning $\dfrac{dx}{dt} = f(x)$, not $f(t,x)$), differential equations approximately follow solutions to $\dfrac{dx}{dt} = \left.\frac{df}{dx}\right\vert_{x^*}(x-x^*)$, where $x(t) = x^*$ is an equilibrium solution and $\left.\frac{df}{dx}\right\vert_{x^*}, x^*$ are constants.  Solutions to $\dfrac{dx}{dt} = \left.\frac{df}{dx}\right\vert_{x^*}(x-x^*)$ display exponential growth or decay (set by the sign of $\left.\frac{df}{dx}\right\vert_{x^*}$).
    \item Numerical approximation methods such as forward Euler can be used to approximate a single solution to an initial value problem, $\dfrac{dx}{dt} = f(t,x), x(0) = x_0$.
\end{itemize}
\end{tcolorbox}

\vspace{0.2cm}
\hrule
\vspace{0.2cm}

\eject

\noindent\textbf{Exact solutions: method of separation of variables} \S 11.4

\begin{tcolorbox}
\begin{itemize}
\itemsep0em
    \item Given a differential equation $\dfrac{dy}{dx} = f(x,y)$, if the differential equation can be rewritten as $\dfrac{dy}{dx} = g(x)h(y)$, it is possible to attempt to use the method of \textbf{separation of variables} to find solution curves to the differential equation.
    \item To use the method:
    \begin{itemize}
        \item Rearrange the equation: $\dfrac{1}{h(y)}\dfrac{dy}{dx} = g(x)$, `separating' the variables. 
        \item Integrate the equation with respect to $x$.  $\displaystyle\int \dfrac{1}{h(y)}\dfrac{dy}{dx}dx = \int g(x)dx$.
        \item Use $u$-substitution for the left hand side with $y = y(x)$. $dy = \frac{dy}{dx}dx$.
        \item If possible, integrate each side of $\displaystyle\int \dfrac{1}{h(y)}dy = \int g(x)dx$.  Often, you will not be able to complete the integrals, but when you are able to, you can at least construct an implicit equation relating $y$ and $x$.
    \end{itemize}
    \item As a shorthand, given $\dfrac{dy}{dx} = g(x)h(y)$, the process of separation of variables is frequently written as $\dfrac{1}{h(y)}dy = g(x)dx$ (treating $dy/dx$ as a fraction).
\end{itemize}
\end{tcolorbox}

\noindent\textbf{Example: using the method.} 

Let $\dfrac{dy}{dx} = -\dfrac{x}{y}$.  Use the method of separation of variables to generate a family of solutions to this differential equation.

\vspace{2in}

\noindent\textbf{Example: does the method apply?}

For each of the following differential equations, identify whether it is separable.

\begin{enumerate}
\itemsep2em
    \item $y' = y$
    \item $y' = \sin(x+y)$
    \item $y' - xy = 0$
    \item $y' = \frac{x+y}{x+2y}$
    \item $y' = 2x$
    \vspace{0.5in}
\end{enumerate}

\noindent\textbf{Example: using the method}

\url{https://www.tandfonline.com/doi/pdf/10.1080/00029890.1998.12004909}

Torricelli's law is a model for how fluid drains from a hole.  $v = \sqrt{2gy}$ where $v$ is the velocity of the fluid, $g$ is the gravitation constant, and $y$ is the height of fluid above the hole.  \emph{Evidently, before Torricelli, it was thought that $v \propto h$.}

Assume of the effective area of the hole is $a$, so $\frac{dV}{dt} = -a\sqrt{2gy}$ where $V$ is the volume of the container that has a hole.

Assume the container has cylindrical walls, so $V = Ay$ where $A$ is the cross-sectional area.  We have $\dfrac{dy}{dt} = -k\sqrt{y}$ where $k = a\sqrt{2g}/A$.
\begin{enumerate}
    \item Find a solution to the initial value problem $\dfrac{dy}{dt} = -k\sqrt{y}, y(0) = y_0>0$.
    \vspace{2in}
    
    \item Identify a range of time over which your solution makes sense.
    \vspace{2in}
    
    \item Consider $y_0 = 0$, so the container is empty at time $0$.  Find a solution where the container emptied at time $t = -1$, one where it emptied at time $0$, and one where it was empty for all time.
    \vspace{2.5in}
\end{enumerate}

\eject

\vspace{0.2cm}
\hrule
\vspace{0.2cm}


\noindent\textbf{Existence and uniqueness of solutions (a comment)}
\begin{tcolorbox}
\begin{itemize}
\itemsep0em
\item The solution to an equation or system of equations is called \textbf{unique} when there is only one solution.  Other possibilities are that there are multiple solutions or no solutions.  \emph{Identifying when systems of linear equations have no solution, one solution, or multiple solutions is a central topic in linear algebra.}
\item An \textbf{existence} and \textbf{uniqueness} theorem for solutions to an initial value problem $\dfrac{dx}{dt} = f(x), x(0) = x_0$ was proven in the 1890s (Picard-Lindelöf theorem).  Existence of a solution (on a finite time interval) can be guaranteed when $f$ is a continuous function (you can draw it without lifting your pen).  Uniqueness requires an additional condition on how $f$ changes with a change in $x$.  When $\frac{df}{dx}$ is continuous as well, there is a unique solution (this condition is tighter than necessary but is sufficient).  
\end{itemize}
\end{tcolorbox}

\vspace{0.2cm}
\hrule
\vspace{0.2cm}

% \noindent\textbf{Exact solutions: using an integrating factor}
% \begin{tcolorbox}
% \begin{itemize}
%     \item For a linear first order differential equation of the form $\dfrac{dx}{dt} + p(t)x = g(t)$, with $p$ and $g$ continuous:
%     \begin{itemize}
%         \item Let $\mu(t)$ be an \textbf{integrating factor}.  Write $\mu(t)\dfrac{dx}{dt} + \mu(t)p(t)y = \mu(t)g(t)$.
%         \item Assume that $\mu(t)$ is chosen so that $\mu(t)p(t) = \frac{d\mu}{dt}$, so we have $\mu\dfrac{dx}{dt} + \dfrac{d\mu}{dt}x = \mu(t)g(t)$.
%         \item Using the product rule: $\dfrac{d}{dt}(\mu x) = \mu(t)g(t)$, so $x =\left(\int \mu g dt + c\right)/\mu$.
%     \end{itemize}
%     \item To find $\mu$, we have $\frac{d\mu}{dt} = \mu p(t)$, a separable differential equation.
% \end{itemize}
% \end{tcolorbox}


% \noindent\textbf{Example: using the method.}

% Find a solution to $\dfrac{dx}{dt} = b - a x$ using
% \begin{enumerate}
%     \item an integrating factor.
%     \vspace{1.8in}
%     \item By separation of variables.
%     \vspace{1in}
% \end{enumerate}

% \vspace{0.2cm}
% \hrule
% \vspace{0.2cm}


% \noindent\textbf{Problem.}  Consider the initial value problem with the logistic differential equation, $\dfrac{dx}{dt} = ax(1-x/k), x(0) = k/2$.
% \begin{enumerate}
%     \item Find a solution to this differential equation.
%     \vspace{2.5in}
    
%     \item By taking a limit, identify the long term behavior of your solution as $t\rightarrow \infty$.  In addition, identify the negative time behavior as $t\rightarrow -\infty$.
% \end{enumerate}

% \vfill

% \vspace{0.2cm}
% \hrule
% \vspace{0.2cm}



\end{document}