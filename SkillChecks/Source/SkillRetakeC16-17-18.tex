\documentclass[12pt,letterpaper,noanswers]{exam}
\usepackage[usenames,dvipsnames,svgnames,table]{xcolor}
\usepackage[margin=0.9in]{geometry}
\renewcommand{\familydefault}{\sfdefault}
\usepackage{multicol}
\pagestyle{head}
\definecolor{c03}{HTML}{FFDDDD}
\header{AM 22b Class 22}{Updated \today.}{Skill Check Retake}
\runningheadrule
\headrule
\usepackage{graphicx} % more modern
\usepackage{amsmath} 
\usepackage{amssymb} 
\usepackage{hyperref}
\usepackage{tcolorbox}
\usepackage[numbered,autolinebreaks,useliterate]{mcode}

\newcommand{\mb}[1]{\underline{#1}}

\begin{document}
 \pdfpageheight 11in 
  \pdfpagewidth 8.5in

% Name: \rule{2.5in}{0.5pt}
% \vspace{4mm}


\begin{questions}
\item Based on the following code, identify (1) the rectangular box being used to enclose the region of integration, $R$, and (2) the region of integration.

\begin{lstlisting}
fc = @(x,y) x.*y;
npoints = 100000;
xyvals = rand(npoints,2);
xyvals(:,1) = xyvals(:,1)*9;
xyvals(:,2) = xyvals(:,2)*3;
indomain = (-xyvals(:,1)+xyvals(:,2).^2)<0; %set to 1 if in domain; 0 if not
sum(fc(xyvals(:,1),xyvals(:,2)).*indomain)*27/npoints
\end{lstlisting}

\vfill

\item (parameterization of a circle). Provide a parameterization for a circle of radius $3$, centered at $(2,-5)$, and traversed clockwise.  Traverse the circle once (at a speed of your choice).
\vfill

\item (velocity) Find $\mb v(t)$ for $\mb{r}(t) = \langle e^t, 3t^4, \cos t\rangle$.

\vfill

\end{questions}

\end{document}